\chapter{Limitations} \label{chap:limitations}

\section{Class Imbalance}
One limitation in multi-organ segmentation is class imbalance. Class imbalance is occurs 
when there are one or more classes that are more frequent occuring than the other classes. 
Simply put, there is a skewness towards the majority class.
For instance, for segmentation of pelvic region, rectum and bladder 
are often different size or position from the femoral heads.
Hence, training a network with class imbalanced data is causing 
unstable segmentation model, which is biases towards the classes 
of different organs. The consequence of class-imbalance is listed below:
\begin{itemize}
        \item No learning due to easy negatives
        \begin{abstract}
                There will be no improvent to the model's performance due to that the model already did decent job overall.
                In this way, the training process of the model becomes inefficient due to the fact that most of the locations 
                in the image are easy negatives, thus contribute no useful learning.
        \end{abstract}
        \item Cumulative effect of many easy negatives
        \begin{abstract}
                When small losses from such easy negative classified classes are summed over many image,
                it exagerates the overall total loss, emphasizing to degenerated models. This tends to shift the model on 
                focusing only on classifiyng negative examples correctly.
        \end{abstract}
\end{itemize}

To alleviate this problem, a right use of loss 
functions is crucial for these tasks.

Proposed loss functions that tackle this problem
\begin{itemize}
        \item Combo loss
        \item Focal Tversky loss
\end{itemize}
All the afforementioned loss functions, are a combination of Focal Loss function. 
Focal loss is an improved loss function that solves a 2-fold problem. 
Firstly, Focal loss gives a balance between easy and hard examples. Secondly, it gives balance between positive and negative examples.
Simply put, the loss function should be able to downweight the easy examples and focus on complex ones.

\begin{equation}
        FL(p_{t})= -(1-p_{t})^\gamma*\log (p_{t}).
\end{equation}






\section{Over-fitting}
Another crucial challenge in multi-organ segmentation is data scarcity. In other words, 
the model has significantly good performance on the training set but weak performance 
on the validation and test set. This could caused due to the lack of available dataset, on the grounds 
that manual segmentation of the organs is deemed a laborious and time-consuming process. 
Furthermore, the wrong (or partially wrong) deliniated dataset produce similar consequences, 
provoking misunderstandings on the trained model. A dominant solution to tackle this issue is data augmentation, however, 
a large dataset is essential for multiorgan segmentation.