%%%%%%%%%%%%%%%%%%%%%%%%%%%%%%%%%%%%%%%%%%%%%%%%%%%%%%%%%%%%%%%%%%%%%%%%%%%
% This is a sample header for a sample dissertation. Fill in the name,
% and the other information. LaTeX will work out the table of
% content, the list of figures and of tables for you.
%%%%%%%%%%%%%%%%%%%%%%%%%%%%%%%%%%%%%%%%%%%%%%%%%%%%%%%%%%%%%%%%%%%%%%%%%%%

\newpage
\thispagestyle{empty}

% ******* Title page *******
% **************************

\vspace*{2cm}
\begin{center}
{\Large\bf MSc. Thesis VIBOT\\} \vspace{2cm} {\large
Emmanouil Koutoulakis\\
\vspace{2cm}
Department of Computer Architecture and Technology \\
University of Burgundy}

\end{center}

\vspace{7cm}
\begin{center}
{\large A Thesis Submitted for the Degree of \\MSc in Vision and Robotics (VIBOT) \\\vspace{0.3cm} $\cdot$ 2022
$\cdot$}
\end{center}
\singlespacing


%ABSTRACT
\begin{abstract}
Magnetic reasonance (MR) images play a vital role in rapid diagnosis and treatment planning. Psysists analyze images to determine
the size of the organ, to plan the therapy and then to observe the evolution of the disease. Unfortunately, this process deemed
a time-consuming and laborious work for the physisians, creating impediments in the fast treatment plannig. Following this incentive, 
the objective of this thesis is to evaluate and propose an efficient learning-based method for automatic segmentation of the organs 
at risk (OARs) in pelvic region in the context of radiotherapy planning, using MR images.
Primarily, we centralized on organ-per-organ and multi-organ segmentation using Convolutional Neural Networks (CNN) 
trained on MR images which are manually segmented by experts. We tested 3 state of the art architectures to select the optimal one. 
The experiments were both 2D and 3D CNNs. Then, we concluded to one segmentation model having the best results and being efficient 
in terms of computational complexity. Due to the lack of enough patients, we implemented a data-augmentation methods to increase 
the size of the training set.

%  -------------
%   1. Metrics
%   2. Refer to dataset limitation
%   3. Brief reference of the models that used
\vspace*{5cm}



\begin{center}
\begin{quote}
\it Research is what I'm doing when I don't know what I'm
doing.\,\ldots
\end{quote}
\end{center}
\hfill{\small Werner von Braun}

\end{abstract}

\doublespacing

%\pagestyle{empty}
\pagenumbering{roman}
\setcounter{page}{1} \pagestyle{plain}


\tableofcontents

\listoffigures
\listoftables

\chapter*{Acknowledgments}
\addcontentsline{toc}{chapter}
         {\protect\numberline{Acknowledgments\hspace{-96pt}}}

Any acknowledgments???

\pagestyle{fancy}