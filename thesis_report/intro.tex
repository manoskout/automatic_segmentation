\chapter{Introduction} \label{chap:intro}
The objective of this chapter is to intuitively explain the thesis from its clinical and technical 
point of view. To begin with, we discuss aspects that are crucial to understanding the clinical
incentives of the proposed approach. We start by defining the pelvic region and organs at risk (OARs), which are
the main focus of this thesis. Then, we present the commonly used types of medical
images, and we discuss the role of segmentation tasks in treatment planning.
Second, we discuss important aspects related to deep learning, atlas-based and image processing techniques, 
which form the basis of most of the current state-of-the-art methods for image segmentation. In particular,
we discuss its main advantages and drawbacks in the segmentation of organs at risk.
In the last phase, we establish our main contributions and the structure of the thesis
\section{Preparing your dissertation} \label{sect:thefirst}

You are strongly encouraged to use the Latex templates provided.

\subsection{Paper}
The manuscript should be in A4 size, and the printed paper should
be of at least 70 gsm.

\subsection{Font and margins}
Thesis should be printed on both sides of the paper. Use no less
than 1.5 spacing, with quotations and notes single-spaced.
Regarding \textbf{Character size}, not less than 2.0mm for
capitals and 1.5mm for x-height (the height of a lower-case x). Us
a serif font (i.e. Times) between 10 and 12 points. Use consistent
and clear fonts through all the document.

% The text layout should be approximately as follows:

% \begin{itemize}
%     \item $4cm$ binding margin
%     \item $2cm$ head margin (top of page)
%     \item $2.5cm$ fore-edge margin
%     \item $4cm$ tail margin (bottom of page)
% \end{itemize}


% \section{Title Page}
% The title page should contain the title of thesis, authors name,
% and at the foot of the page: the name of degree,  Your University,
% and the year of presentation. Something like this:

% \vspace*{1cm}
% \begin{center}
% {\Large\bf MSc. Thesis VIBOT\\} \vspace{2cm} {\large
% Emmanouil Koutoulakis\'i\\
% \vspace{1cm}
% Department of Computer Architecture and Technology \\
% University of Burgundy}

% \end{center}

% \vspace{2cm}
% \begin{center}
% {\large A Thesis Submitted for the Degree of MSc in
% Vision and Robotics (VIBOT)\\ \vspace{0.3cm} $\cdot$ 2022 $\cdot$}
% \end{center}


% \subsection{References}
% You can reference other authors by using the $cite command$
% \cite{Pokorski:1998hr}. You are encouraged to use bib files and
% let bibtex do the job for you.